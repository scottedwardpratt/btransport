\documentclass[aps, prc, 12pt, nofootinbib, showpacs, superscriptaddress, tightenlines, groupedaddress]{revtex4-2}
\usepackage{amsmath,amssymb,amsbsy,bm}
\usepackage{graphicx}
\usepackage{comment}
\usepackage{float}
\usepackage[colorlinks=true,linkcolor=blue,citecolor=blue,urlcolor=blue]{hyperref}
\usepackage[margin=0.75in]{geometry}
\usepackage{silence}
\WarningFilter{revtex4-2}{Repair the float}

%\usepackage[table]{xcolor}

\begin{document}

\title{Baryon Transport in Color Flux Tubes}
\author{Scott Pratt}
\affiliation{Department of Physics and Astronomy and National Superconducting Cyclotron Laboratory\\
Michigan State University, East Lansing, MI 48824~~USA}
\date{\today}

\pacs{}

\begin{abstract}
baryon transport ...
\end{abstract}

\maketitle

\section{Introduction}

Recent observations in proton-proton, proton-nucleus and nucleus collisions have inspired interest in baryon transport in hadronic collisions. Baryons from the initial incoming nuclei tend to lose roughly one unit of rapidity in the collision. The distribution appears to fall off exponentially as a function of rapidity loss, a result qualitatively consistent with the baryon-junction hypothesis, where baryon number can be carried by gluonic degrees of freedom rather than solely by quarks. Naturally, this inspires one to re-examine less exotic pictures of baryon transport. In particular, flux-tube or string models have long been a staple of the stopping of baryons in hadronic collisions. The idea is that the baryons, after exchanging a gluon, develop a tube of electric flux that provides a force, or string tension acting on the baryon, thus slowing it down. At some point the string or tube breaks, resulting in the baryon losing some momentum given by the string tension multiplied by the time until the tube ruptures, or the string breaks. Another model of the initial stage, the color-glass condensate is also predicated on a picture of initially longitudinal color electric fields, a picture which shares some of the same physical motivations as the flux-tube and string pictures.

Here, the role of baryon transport in color fields is investigated within the context of the flux-tube picture. This approach is largely heuristic, but nonetheless might demystify the exponential fall-off of the stopping. A second goal is to better understand how color fields, which couple to color, not baryon number, induce baryon currents and baryon number polarization, which result in baryon number traversing large swaths of rapidity. 

The following sections investigate three separate questions. The next section considers the simplest version of a flux tube, one with a quark at one end and an antiquark on the other. After reviewing how even in such a simple string the baryon number at the end of the strings vanishes into mesons, the merging of two simple strings into one is considered. It is shown that when two tubes from the two quarks in the projectile merge into one, with the new string connecting to a quark in the target hadron, that the baryon number from the projectile is transferred to the point of merging. Further, it is shown how this can be driven by simply exchanging a soft gluon between the target and projectile. Finally, it is pointed out that the most likely distribution of rapidity range of the baryon movement is exponential, assuming the paths are thermally weighted.

Section \ref{sec:kubo} is more formal, and describes how one can write a linear response theory connecting both baryon current and polarization to a field-like quantity. The quantities that play the role of fields are shown to be gauge-invariant correlators between the gluonic field and the color charge driving the field. Given that this physics is non-perturbative, linear response theory is questionable. Nonetheless, the exercise provides a basis for understanding how baryonic transport is driven by color fields. Even purely gluonic excitations are shown to have the ability to induce baryon currents.

A given point in a single flux tube can be characterized by the color multiplet of the charge to each side of the point. Section \ref{sec:compound} reviews how a tube with characterized by a higher multiplet might decay to color singlets, and how such decays can induce baryon transport across the tube. The potential observable consequences of these pictures are discussed in the summary. Observables are proposed, but given the qualitative and highly heuristic nature of the flux-tube paradigm, further study is needed before understanding whether the proposed measurements  might be sufficiently strong to warrant investing experimental effort. 

\section{Baryon Transport from Merging Simple Flux Tubes}\label{sec:simple}

Here, we consider the merging of two simple flux tubes into one. The term ``simple'' refers to the fact that the tube is generated by a single quark or anti-quark. In a simple tube, one can bisect the tube at any point and classify the color charge on each side of the bisection by its color multiplet, $p,q$. In SU(3), the multiplet is defined by two integers, which differs from SU(2), where a multiplet is denoted by a single number, $j$, e.g. the total angular momentum. Graphically, the state can be enumerated by the graphs in Fig. \ref{fig:pqmultiplet}. Multiplets denoted by $(p_1,q_1)$ and $(p_2,q_2)$ can be combined to create multiplets denoted by $(p',q')$. In SU(2), multiplets denoted by $j_1$ and $j_2$ only couple to singlet $j'=0$ if $j_1=j_2$. Similarly, for SU(3) two multiplets couple to a color singlet only if $p_1=q_2$ and $p_2=q_1$. For example, the $(1,0)$ multiplet (quark) and the $(0,1)$ multiplet (antiquark) can couple to a color singlet.
\begin{figure}
\centerline{\includegraphics[width=0.6\textwidth]{figs/pqmultiplet.pdf}}
\caption{\label{fig:pqmultiplet}
Graphical representation of $(p,q)$ multiplets. The quark and antiquark color{} triplets are represented by $(1,0)$ and $(0,1)$, respectively, while the gluon octet is $(1,1)$. Each dot represents one projection of the multiplet, if the dot is on the outer ring. Subsequently, for the next inner-ring ring a dot represents two states, or three states for the next inner-ring, although the increasing degeneracy stops once a ring has either $p=0$ or $q=0$. The net degeneracy of the multiplet is $(p+1)(q+1)(p+q+2)/2$. In SU(2) one can combine multiplets of $j_1$ and $j_2$ to form several multiplets $j'$. Similarly, in SU(3) one can combine multiplets of $(p_1,q_1)$ and $(p_2,q_w)$ to form multiplets of several combinations of $(p',q')$.
}
\end{figure}

Before addressing merging, we review the case of a simple flux tube between a quark and antiquark. Figure \ref{fig:simpletube} illustrates how the flux can be reduced by quark-antiquark pairs tunneling out of the vacuum. One can pick any position along the tube (denoted by a dahed line in the figure), and after the tube has decayed, via quark-antiquark pair production, all matter to either side of the position should return to a color singlet. This is most easily accomplished by creating a $q\bar{q}$ pair out of the vacuum, with the antiquark moving to the side of the tube that had initially had a single quark, and the quark moving to the opposite side. Creating the pair required energy, but by separating sufficient distance the reduced field energy can more than account for the masses and kinetic energies of the quark-antiquark pair. If the string tension, and energy per unit length, is $\lambda$, the field energy gained by producing the pair is $\lambda \ell$, where $\ell$ is the distance separating the quark and the antiquark. This process is sometimes compared to Schwinger pair production \cite{Schwinger}, where $e^+e^-$ pairs can be created in sufficiently strong electric fields. The QCD process differs in that the flux tube energy, $\lambda\ell$, assists with the pair production, whereas in the constant electric field case, one only has $eE\ell$, where the electric field is $E$. The Schwinger effect has been proposed to exist in regions of extremely high electric field in plasmas \cite{Schwinger_plasma}, in graphene \cite{Schwinger_graphene}, where the electrons and holes become nearly massless, as well as in the context of hadronic collisions \cite{Wong:1994ei,Suganuma:1991ha}.
\begin{figure}
\centerline{\includegraphics[width=0.4\textwidth]{figs/simpletube.pdf}}
\caption{\label{fig:simpletube}
A simple flux tube between a quark $(p=1,q=0)$ and an antiquark $(p=0,q=1)$ is illustrated in the upper panel, with the lighter and darker circles representing the quark and antiquark respectively. Drawing a vertical line through a flux tube divides space into two regions, the left side of the dashed line in a color multiplet $(p=1,q=0)$ and the right side in the multiplet $(p=0,q=1)$. The lower panel illustrates how quark-antiquark pairs can tunnel out of the vacuum so that the space between the tunneling quarks is free of color fields. Tunneling quark-antiquark pairs are denoted by the arrows.  Dividing space that does not bisect a tube results in both sides being in color singlets.
}
\end{figure}

A simple flux tube, with a quark at one end and an antiquark on the other, typically decays completely into mesons. Even though each end of the string has baryon number, and even though there are no baryons in the final state, no quark moved any significant distance along the tube. Instead, quark-antiquark dipoles appeared, effectively shifting a line of quarks in one direction and a line of antiquarks in the opposite direction. The shift represents a current. This is similar to placing a dielectric in an electric field. Opposite charge densities appear on each side of the dielectric even though no individual charge moved further than the size of a dipole. The baryon current involved can be equated to the time rate of change of the polarization. One way in which a color flux tube differs from electric field is that the induced current will persist until the field is completely neutralized, unlike in a dielectric, where the current persists until the induced polarization reduces the initial field by the permeability of the matter.

\begin{figure}
\centerline{\includegraphics[width=0.48\textwidth]{figs/simplemerge.pdf}}
\centerline{\includegraphics[width=0.6\textwidth]{figs/gluonexchange.pdf}}
\caption{\label{fig:merge}
Upper illustration: The merge of two tubes into one, where the left-hand side tubes begin with two quarks, and the right-hand side ends with one quark. The two quark multiplets couple to a $(0,1)$ multiplet, which then matches with the quark multiplet on the right. The baryon number, is then located at the point where the two tubes merged. This represents a situation where one quark is scattered far, from its original momentum, and the flux tubes must adjust to form local color singlets. The original quark pair must couple to $(p=0,q=1)$ if the overall configuration is to be a color singlet.\\
Lower illustration: This represents the case where two baryons exchange a soft gluon. In that case both of the baryons are transformed into $(p=1,q=1)$ multiplets. Assuming that for each baryon there is a diquark with $(p=0,q=1)$, the overall color singlet is restored by migrating the baryon number to the two places where the strings merge.\\
Darker circles represent quarks, while darker circles represent antiquarks. Ovals represent fluxtubes, and arrows describe quark-antiquark pairs that tunneled from vacuum.
}
\end{figure}
Figure \ref{fig:merge} shows how baryon number can move from the target or projectile rapidity toward central rapidity through merging of simple flux tubes. As was the case for a single tube with a quark at the end, quark-antiquark pairs appear in a polarized manner. The baryon number of 1/3 at the end of the string then gets absorbed into mesons. Following the string, once the mesons appear, there is a quark left over on the string. This leftover quark represents the baryon transport. If three strings merge, the three missing quarks can produce a baryon. However, if the three quarks originated from the same baryon, e.g. a projectile proton, and if that proton exchanged a gluon with the target, then those three quarks would then be in a color octet, $(p=1,q=1)$, and they could not form a baryon. However, if one of the three merging tubes came from the opposite side, then a baryon can form at the merging point, as illustrated in the upper diagram of Fig. \ref{fig:merge}.

The lower diagram of Fig. \ref{fig:merge} illustrates how all six quarks can recombine through string merging. Thus, baryon number should move in from both directions. Of course, the most energetically favorable path is for the tubes to merge as quickly as possible. When two quarks merge into a single tube, and if the two quarks were coupled to a $(p=0,q=1)$ state, that is known as a diquark. Once merged the string should have the same field energy as that between a quark and an antiquark. If system chooses the point to merge based on local thermal penalties, one would expect the probability for merging to fall exponentially with the separation from the initial rapidity of the baryon. Phenomenologically, one expects approximately one tube breaking per unit of rapidity. If the characteristic energy of the string is similar to the characteristic temperature, one might expect the characteristic exponential falloff to of order unity, but it should be emphasized that it would not be surprising if the actual exponential scale was a half unit of rapidity. The chance that a baryon might be transported from the target or projectile rapidity, a difference of 7 units at the LHC, would be extremely sensitive to this scale.

Figure \ref{fig:merge} illustrates two tubes merging that began at the target or projectile nucleons. One might also consider such tubes splitting again, as one moves further towards mid-rapidity, perhaps splitting and merging more than once. In that case if a tube splits and merges, a baryon would be at one end and an antibaryon at the other. There would be no preference for baryon to appear before or after the anti-baryon. If this were the mechanism for how baryon-antibaryon pairs are created from flux tubes, one would expect that the balancing pair's separation in rapidity to fall off exponentially, with the same exponential falloff as that characterizing the transport of baryon number from the target or projectile. Observing such separation would require experimental particle identification across a few units of rapidity.

In addition to providing additional baryon number, if the projectile and target are both protons, they also provide additional electric charge. In a flux tube the dipole-like tunneling quarks line up with the antiquark on the side of the quark at which the tube originated. However, for electric charge, the tunneling pair would roughly have equal probability of being a $u\bar{u},~d\bar{d}$ or $s\bar{s}$ pair. This would result in zero polarization of the electric charge. If the $s\bar{s}$ pairs were slightly less, then a small electric charge transport would ensue. It would be smaller in magnitude than that for baryon number, but be characterized by the same exponential cutoff. 

\section{Connecting Baryon Transport to Color Fields}\label{sec:kubo}

It is not obvious to see how baryon transport is connected to gluonic fields, $G^{\mu\nu}_a(x)$. Unlike the case for electric fields, gluon fields have a color index, $a$, and because all of nature is in a color singlet, the expectation of the field operators are zero, $\langle G^{\mu\nu}_a(x)\rangle=0$. Therefore, one must describe the fields through operators involving products of operators with color indices, i.e. corrrelators involving two or more operators with color indices. Further, gluons couple to color charge, not to baryon charge. The rather modest goal of this section is to write operators in terms of QCD field operators that represent the baryon polarization or baryon current and the the field-like operators that drive the polarization or current. These operators will be discussed in the context of flux tubes. None of the expressions derived here will be applied to quantitatively predict observables in this study. However, the presentation may help elucidate how fields in QCD can inspire baryon movement and separation.

\subsection{Color-Neutral Correlators}

Here, the color electric/magnetic field operators and the color current operators will be considered, $G^{\mu\nu}_a(x)$, and $j^\mu_a(x)$. Instead of the usual definition of such operators, we add an additional feature that enables the construction of gauge-invariant correlators. For any operator $\mathcal{O}(x)$, an additional attached operator is inferred \cite{Elze:1989un},
\begin{eqnarray}
\mathcal{O}_a(x)&\rightarrow \exp\{i\mathcal{P}\int d\vec{\ell}\cdot\vec{A}\}\mathcal{O}_a(x)\mathcal{P}\exp\{ig\int d\vec{\ell}\cdot\vec{A}\}.
\end{eqnarray}
Here, $\mathcal{P}$ is the path-ordering operator. With this definition, if some correlator $\langle\mathcal{O}_a(x)\mathcal{O}_a(x)\rangle$ is gauge invariant, then $\langle\mathcal{O}_a(x)\mathcal{O}_a(y)\rangle$ is also gauge invariant, though one needs to realize that the new correlator might depend on the actual path chosen. With this definition, one can write expressions that appear similar to those for Maxwell's equations,
\begin{eqnarray}
\partial_\mu j^\mu_a(x)&=&0,\\
\nonumber
\partial_\mu G^{\mu\nu}_a(x)&=&j^\nu_a(x).
\end{eqnarray}
Because all observables must be invariant to rotations in color space, one knows that $\langle\mathcal{O}_a\rangle=0$ and one must consider only colorless correlators. To find quantities which can be treated, at least to some degree, similarly as fields, we consider the following operators defined in terms of operators $j_a^\mu(x)$, where $a$ is a color index. 
\begin{eqnarray}
Q_{a,\Omega}&=&\int d\Omega_\mu j^{\mu}_a(x),\\
\nonumber
\langle Q^{(2)}_{\Omega_1,\Omega_2}\rangle&=&\int d\Omega_{1,\mu}d\Omega_{2,\nu}\langle j^\mu_a(x_1) j^\nu_a(x_2)\rangle,\\
\nonumber
&=&\langle Q_{a,\Omega_1}Q_{a,\Omega_2}\rangle,\\
\nonumber
\langle j^{(2)\nu}_{\Omega_1}(x)\rangle &=&\int d\Omega_{1,\mu} \langle j^\mu_a(x_1) j^\nu_a(x)\rangle\\
\nonumber
&=&\langle Q_{a,\Omega_1}j^\nu_a(x)\rangle.
\end{eqnarray}
Here, the superscripts $(2)$ reference the fact that these correlations are quadratic in the color charge. The volumes $\Omega_1$ and $\Omega_2$ can be any chosen volumes, and might be the same volume. If the volumes cover all space, and if the system is in an overall color singlet, $Q^{(2)}=0$. One can also define correlators using the fields,
\begin{eqnarray}
\langle G^{(2)\mu\nu}_{\Omega}(x)\rangle&=&\int d\Omega\langle Q_{a,\Omega}G^{\mu\nu}_a(x)\rangle.
\end{eqnarray}

As mentioned above, the correlators $\langle\mathcal{O}_a(x)\mathcal{P}_a(x)\rangle$ are gauge-invariant for any operators $\mathcal{O}$ or $\mathcal{P}$ that transform as the color charge, or equivalently as the 8 generators in SU(3). In fact, there are other combinations of operators that transform as color singlets. SU(3) has two Casimir operators, a Casimir being some combination of the group generators that commutes with the eight generators, $\lambda_a$. The first Casimir, often referred to as the quadratic Casimir
\begin{eqnarray}
C^{\rm(quadratic)}&=&\sum_a\lambda_a\lambda_a,
\end{eqnarray}
corresponds to the two-body correlators above. Unlike SU(2), SU(3) has a second Casimir, the cubic Casimir,
\begin{eqnarray}
C^{\rm(cubic)}&=&\sum_{abc}d_{abc}\lambda_a\lambda_b\lambda_c,
\end{eqnarray}
where $d_{abc}$ are the symmetric structure constants.
\begin{eqnarray}
\{\lambda_a,\lambda_b\}&=&\frac{4}{3}\delta_{ab}+d_{abc}\lambda_c.
\end{eqnarray}
It is this second Casimir that will be associated with baryon transport. Once can define a second set of current and charge-like operators, based on the cubic Casimir,
\begin{eqnarray}
\langle j^{(3)\mu}_{\Omega_1,\Omega_2}(x)\rangle &=&d_{abc}\langle Q_{a,\Omega_1}Q_{b,\Omega_1}j^\mu_c(x)\rangle,\\
\nonumber
\langle Q^{(3)}_{\Omega_1,\Omega_2,\Omega_3}\rangle&=&d_{abc}\int d\Omega_{1,\mu}d\Omega_{2,\nu}d\Omega_{3,\eta}\langle j^\mu_a(x_1) j^\nu_b(x_2)j^\eta_c(x_3)\rangle\\
\nonumber
&=&d_{abc}\langle Q_{a,\Omega_1}Q_{b,\Omega_2},Q_{c,\Omega_3}\rangle.
\end{eqnarray}
Again, the three volumes might or might not be the same. One can define the corresponding correlations involving field operators,
\begin{eqnarray}
\langle G^{(3)\mu\nu}_{\Omega_1,\Omega_2}(x)\rangle&=&d_{abc}\langle Q_{a,\Omega_1}Q_{b,\Omega_2}
G^{\mu\nu}_c(x)\rangle.
\end{eqnarray}

For the purposes of this discussion, two vector quantities are defined,
\begin{eqnarray}
E^{(2)}_{\Omega,i}(x)&=&\langle G^{(2)0i}_{\Omega}(x)\rangle,\\
\nonumber
E^{(3)}_{\Omega_1,\Omega_2,i}(x)&=&\langle G^{(3)0i}_{\Omega_1,\Omega_2}(x)\rangle.
\end{eqnarray}
These represent two measures of the color electric field, but are in fact correlators. They have no color index, as their color is defined relative to the color of the matter within the volume $\Omega$. These spatial vectors participate in two versions of Gauss's Law,
\begin{eqnarray}\label{eq:gauss}
 \oint d\vec{S}_2\cdot\vec{E}^{(2)}_{\Omega_1}(x_2)&=&Q^{(2)}_{\Omega_1,\Omega_2},\\
 \nonumber
 \oint d\vec{S}_3\cdot\vec{E}^{(3)}_{\Omega_1,\Omega_2}(x_3)&=&Q^{(3)}_{\Omega_1,\Omega_2,\Omega_3}.
\end{eqnarray}
Again, the volumes, $\Omega_1,\Omega_2,\Omega_3$, might or might not be chosen as being the same.

For a flux tube, where the fields are constant across a cross section $A$, one can define a volume $\Omega$ as all tube to the left of some position where we bisect the tube. If the charge with that volume are $Q^{(2)}_{\Omega,\Omega}$, Gauss's law states that the field is 
\begin{eqnarray}
E^{(2)}_{\Omega,\Omega}&=&\frac{1}{A}Q^{(2)}_{\Omega,\Omega}\\
\nonumber
&=&\langle Q_{a,\Omega}E_a\rangle.
\end{eqnarray}
For a single charge $a$, $\langle Q_{a,\Omega}Q_{a,\Omega}\rangle=\langle Q^{(2)}_{\Omega,\Omega}\rangle$. Thus, one can relate the energy density to the fields,
\begin{eqnarray}
 \epsilon&=&\frac{1}{2}\langle E_a^2\rangle\\
 \nonumber
 &=&\frac{1}{2}\frac{(E^{(2)}_{\Omega,\Omega})^2}{Q^{(2)}_{\Omega,\Omega}}.
\end{eqnarray}
The energy per unit length, or string tension is then
\begin{eqnarray}\label{eq:tension}
\frac{E}{L}&=&\frac{A}{2}\frac{(E^{(2)}_{\Omega,\Omega})^2}{Q^{(2)}_{\Omega,\Omega}}\\
\nonumber
&=&\frac{1}{2A}Q^{(2)}_{\Omega,\Omega}.
\end{eqnarray}
Thus, $Q^{(2)}$ is a measure of the energy per unit length of the tube, although if the area changes with the strength of the field, the correspondence is not purely linear.

At the end of this section it will be shown how $Q^{(3)}$ is related to baryon number. The meaning of the two charges are, roughly, that $Q^{(2)}$ describes the energy per unit length of the flux tube and $Q^{(3)}$ represents the propensity of the system to attract baryons vs. antibaryons.

\subsection{Kubo Relations for Conductivity and Polarizability}

Here, we consider a system divided into two volumes, $\Omega$ and a small adjacent volume $\delta\Omega$. We will consider the currents and polarizability in $\delta\Omega$, while using $\Omega$ to define the correlators. The field, $\vec{E}^{(3)}(\vec{r}\in\delta\Omega)$ and the current $\vec{j}^{(3)}(\vec{r}\in\delta\Omega)$ will be defined by the charge distribution in $\Omega$. First, to derive the Kubo relation for the conductivity, we consider the interaction with an color field,
\begin{eqnarray}
H^{\rm(int)}&=&-\int d^3r \rho_a(\vec{r})\vec{r}\cdot\vec{E}_a.
\end{eqnarray}
Following the usual steps for deriving the first-order perturbation correction to the current,
\begin{eqnarray}
\langle\vec{j}^{(3)}_{\Omega\Omega}(\vec{r})\rangle&=&-i\int d^3r'\int_{-\infty}^0 dt~\langle[\vec{j}^{(3)}(\vec{r}),
\int d^3r'~\rho_a(\vec{r}',t)\vec{r}'\cdot\vec{E}_a(\vec{r}',t)]\rangle\\
\nonumber
&=&id_{abc}\int d^3r'\int_{-\infty}^0 dt'~\langle[Q_{a,\Omega}Q_{b,\Omega}\vec{j}_c(\vec{r}),
\int d^3r'~\rho_d(\vec{r}',t')\vec{r}'\cdot\vec{E}_d(\vec{r}',t')]\rangle.
\end{eqnarray}
Here, $\vec{r}$ is inside $\delta\Omega$. Whereas the usual step in deriving Kubo relations is to treat the field as a constant external applied field, here the operator $\vec{E}_{\Omega,\Omega}^{(3)}(\vec{r}')$ is factored out instead. Only the $c=d$ term will contribute in this way.
\begin{eqnarray}
\langle\vec{j}^{(3)}_{\Omega\Omega}(\vec{r})\rangle&=&i\int_{-\infty}^0dt'~\int d^3r'\vec{r}'\cdot
\langle[\vec{j}_a(\vec{r}),\rho_a(\vec{r}',t')]\rangle  \vec{r}'\cdot\vec{E}^{(3)}_{\Omega,\Omega}({\rm in}~\delta\Omega).
\end{eqnarray}
Aside from the color indices, the prefactor to the electric field is the same as for calculating the Kubo relation for a normal electromagnetic field. This yields the result,
\begin{eqnarray}\label{eq:kubo}
\langle\vec{j}^{(3)}_{\Omega\Omega}(\vec{r})\rangle&=&\sigma \vec{E}^{(3)}_{\Omega,\Omega}(\vec{r}),\\
\nonumber
\sigma(\vec{r})&=&\frac{1}{2T}\int_{-\infty}^\infty dt'\int d^3r'~\langle\{\vec{j}_{ax}(\vec{r},t=0),\vec{j}_{ax}(\vec{r}',t')\}\rangle.
\end{eqnarray}
Thus, the conductivity, $\sigma$, doesn't include any mention of $\Omega$, and only depends on the local properties. It should be emphasized that Eq. (\ref{eq:jsigmaE}) one describes the contribution to the current due to the charge distribution in $\Omega$. If $\Omega$ is a subvolume, then other regions not only contribute, but their charges might interfere, either constructively or destructively, with those in $\Omega$.

If quarks and gluons are free, one can have currents and fields as presented above. However, if the world is comprised of color singlets, there is no current, but ther can be polarization. As will be discussed in the next subsection, the current $\vec{j}^{(3)}$ is correlated to a baryon current. Similarly, one can define a polarization of the color fields, that is also correlated to the polarization of baryon charge. Here, operators that play the role of polarization, and are also based on the cubic Casimir, are presented. 

The polarization operator is defined here as
\begin{eqnarray}
\vec{\mathcal{P}}_{\Omega_1,\Omega_2}^{(3)}(\vec{r})=\vec{r}d_{abc}\langle Q_{a,\Omega_1}Q_{b,\Omega_2}\rho_c(\vec{r})\rangle.
\end{eqnarray}
Using the same interaction as was used for the calculation of the conductivity above, one can calculate the thermal expectation of the polarization. Setting $\Omega_1=\Omega_2=\Omega$, and factoring out $E^{(3)}$ as was done earlier, one finds
\begin{eqnarray}\label{eq:polarization}
\langle\vec{\mathcal{P}}_{\Omega,\Omega}^{(3)}(\vec{r})\rangle&=&\kappa \langle\vec{E}^{(3)}_{\Omega,\Omega}(\vec{r})\rangle,\\
\nonumber
\kappa(\vec{r})&=&\frac{1}{3T}\int d^3r'\langle \rho_a(\vec{r})[\vec{r}\cdot\vec{r}']\rho_a(\vec{r}')\rangle.
\end{eqnarray}
As was the case for the conductivity, this expression for the polarizability does not depend on the the volume $\Omega$. If the charges appear in uncorrelated $\pm$ pairs, the polarizability becomes $g^2\langle(x_+-x_-)^2\rangle\rho_d$, where $\rho_d$ is the density of dipoles and $x_+-x_-$ is the distance between the two charges, $\pm g$. 



\subsection{Relation to Baryon Transport}

As was promised previoisl, the goal of this subsection is to demonstrate how baryon transport is related to the field $\vec{E}^{(3)}$, which was in turn related to the cubic Casimir. The quadratic and cubic Casimirs can be expressed in terms of the color multiplet labels $p$ and $q$ through
\begin{eqnarray}\label{eq:Q2Q3vspq}
C^{\rm quadratic}&=&(p^2+q^2+3p+3q+pq)/3,\\
C^{\rm cubic}&=&(p-q)(2p+q+3)(2q+p+3)/18.
\end{eqnarray}
The correlators $Q^{(2)}$ and $Q^{(2)}$ are given by the Casimirs,
\begin{eqnarray}
\langle p,q|Q^{(2)}_{\Omega,\Omega}|p,q\rangle&=&g^2C^{\rm quadratic},\\
\langle p,q|Q^{(3)}_{\Omega,\Omega,\Omega}|p,q\rangle&=&g^3C^{\rm cubic},\\
\end{eqnarray}
where $g$ is the couling constant, which plays the role of the fundamental charge, the equivalent of $e$ in electromagnetism. For a quark, $(p=1,q=0)$, one finds that $\langle Q^{(2)}\rangle=4/3$ and $\langle Q^{(3)}\rangle=(10/9)g^3$. An antiquark yields the opposite expectation,  $\langle Q^{(3)}\rangle=-(10/9)g^3$.

The multiplet labels $p$ and $q$ and the charge labels $Q^{(2)}$ and $Q^{(3)}$ represent alternative means to identify the multiplet. The labels $p$ and $q$ can be uniquely expressed in terms of the charges by solving a cubic equation. Defining
\begin{eqnarray}
c&\equiv&9\langle Q^{(2)}/g^2\rangle+9,\\
\nonumber
d&\equiv&18\langle Q^{(3)}/g^3\rangle,\\
\nonumber
\alpha&=&-2\pi /3+\frac{1}{3}\cos^{-1}[(3d/2c)\sqrt{3/c}],\\
\nonumber
y&=&2\sqrt{c/3}\cos\alpha,\\
\nonumber
x&=&-2+\sqrt{4+\langle Q^{(2)}/g^2\rangle-y^2/3},\\
\nonumber
p&=&(x+y)/2,\\
\nonumber
q&=&(x-y)/2.
\end{eqnarray}
The other two solutions to the cubic equation result in either $p$ or $q$ being negative. The solutions are unique in that no two combinations of  $Q^{(2)}$ and $Q^{(3)}$ result in the same $p$ and $q$.

Given that a quark has $p=1,q=0$ and an antiquark has $p=0,q=1$, it is natural to expect that a multiplet with $p>q$ will more likely dissolve into quarks, and that a multiplet with $q>p$ would more likely result in antiquarks. Given that $Q^{(3)}$ is an odd function of $p-q$ and that $Q^{(2)}$ is an even function, any correlation with baryon number is through the correlation with $Q^{(3)}$. Correspondingly, the field $E^{(2)}$ does not drive baryon transport, but $E^{(3)}$ does.

To express baryon transport, one must first understand $d\langle B\rangle/d\langle Q^{(3)}\rangle$. Once this is determined, the baryon current is
\begin{eqnarray}
\vec{j}_B(\vec{r})&=&\frac{d\langle B_{\delta\Omega}\rangle}{dQ^{(3)}_{\Omega,\Omega,\delta\Omega}}\sigma \vec{E}^{(3)}_{\Omega,\Omega}(\vec{r}).
\end{eqnarray}
Here, $\vec{r}\in\delta\Omega$. This represents the current inside the $\delta\Omega$ driven by the charge in $\Omega$. Statistical considerations give
\begin{eqnarray}
\frac{d\langle B_{\delta\Omega}\rangle}{dQ^{(3)}_{\Omega,\Omega,\delta\Omega}}&=&
\frac{\langle B_{\delta\Omega} Q^{(3)}_{\Omega,\Omega,\delta\Omega}\rangle}
{\langle [Q^{(3)}_{\Omega,\Omega,\delta\Omega}]^2  \rangle}.
\end{eqnarray}

One can follow the same arguments to calculate the baryon polarizability operator,
\begin{eqnarray}
\vec{P}_B(\vec{r})&=&\vec{r}d_{abc}\langle \rho_B(\vec{r})\rangle.
\end{eqnarray}
The ratio of $P_B$ to $P^{(3)}$ is the same as would be used for the current above. The baryon current and polarizability are then
\begin{eqnarray}
\langle\vec{j}_B(\vec{r})\rangle&=&\frac{d\langle B_{\delta\Omega}\rangle}{dQ^{(3)}_{\Omega,\Omega,\delta\Omega}}\sigma
\langle\vec{E}^{(3)}_{\Omega,\Omega}(\vec{r})\rangle,\\
\nonumber
\langle\vec{P}_B(\vec{r})\rangle&=&\frac{d\langle B_{\delta\Omega}\rangle}{dQ^{(3)}_{\Omega,\Omega,\delta\Omega}}\kappa
\langle\vec{E}^{(3)}_{\Omega,\Omega}(\vec{r})\rangle.
\end{eqnarray}

\subsection{Scaled Correlators}

The operators $Q^{(2)}, Q^{(3)},E^{(2)}(\vec{r})$ and $E^{(3)}(\vec{r})$ are correlators. If one assigns a dimension to charge coupling $g$, the units are $g^2$, $g^3$, $g^2/L^2$ and $g^3/L^3$, where $L$ refers to units of length. The operator $Q^{(2)}_{\Omega,\Omega}$ represents a fluctuation, and if one has a bulk system with independent color charges, and no long-range correlation, it would scale with the volume $\Omega$. If one were to consider $Q^{(3)}_{\Omega,\Omega,\delta\Omega}$ and the two field-like operators $E^{(2)}_{\Omega}(\vec{r})$ and $E^{(3)}_{\Omega,\Omega}(\vec{r})$, one would see that for independent charges, the fluctuations of these operators would scale, in the large $\Omega$ limit, as
\begin{eqnarray}
\langle [E^{(2)}_{\Omega}(\vec{r})]^2\rangle\propto \Omega,\\
\nonumber
\langle [Q^{(3)}_{\Omega,\Omega,\delta\Omega}]^2\rangle \propto \Omega^2\delta\Omega,\\
\nonumber
\langle [E^{(3)}_{\Omega,\Omega}(\vec{r})]^2\rangle\propto \Omega^2.
\end{eqnarray}
The correlator with baryon number would scale as
\begin{eqnarray}
\langle Q^{(3)}_{\Omega,\Omega,\delta\Omega}B_{\delta\Omega}\rangle\propto \Omega\delta\Omega.
\end{eqnarray}
Thus, if one were to consider a long flux tube, with $\Omega$ defined as everything to one side, the fluctuation of the field operators at the center of the tube would increase with the size of the tube,and the ratio of $d\langle B\rangle/d\langle Q^{(3)}\rangle$ would fall with size of the tube. 

One can scale the three operators above so that their dimensions and scaling with size are in line with the charge and field operators with which we are more accustomed.
\begin{eqnarray}
\tilde{\vec{E}}^{(2)}_{\Omega}(\vec{r})&\equiv&\frac{\tilde{\vec{E}}^{(2)}_{\Omega}(\vec{r})}{\langle Q^{(2)}_{\Omega,\Omega}\rangle^{1/2}},\\
\nonumber
\tilde{\rho}^{(3)}_{\Omega,\Omega}(\vec{r})&\equiv&\frac{\rho^{(3)}_{\Omega,\Omega}(\vec{r})}{\langle Q^{(2)}_{\Omega,\Omega}\rangle},\\
\nonumber
\vec{\tilde{\mathcal{P}}}_{\Omega,\Omega}^{(3)}(\vec{r})&\equiv&\frac{\vec{\tilde{\mathcal{P}}}_{\Omega_1,\Omega_2}^{(3)}}{\langle Q^{(2)}_{\Omega,\Omega}\rangle}\\
\nonumber
\tilde{Q}^{(3)}_{\Omega,\Omega,\delta\Omega}&\equiv&\frac{Q^{(3)}_{\Omega,\Omega,\delta\Omega}}{\langle Q^{(2)}_{\Omega,\Omega}\rangle},\\
\nonumber
\tilde{\vec{E}}^{(3)}_{\Omega,\Omega}(\vec{r})&\equiv&\frac{\vec{E}^{(3)}_{\Omega,\Omega}(\vec{r})}{\langle Q^{(2)}_{\Omega,\Omega}\rangle}.
\end{eqnarray}
With these definitions, $\tilde{Q}^{(3)}$ has dimensions of charge and $\tilde{E}^{(2)}$ and $\tilde{E}^{(3)}$ both have dimensions of charge per distance. The energy density is then more in line with $|\tilde{E}_{\Omega}^{(2)}(\vec{r})|^2/2$. Again, this would be, roughly, the energy density arising from the charges within $\Omega$. Correlators using any of the quantities scaled as above would then no longer grow with increasing $\Omega$ for systems with no long-range correlations. The charge correlators involving two operators labeled by $\delta\Omega$ would scale with $\delta\Omega$. 

Gauss's law relating  the scaled correlator $\tilde{E}^{(3)}$ to $\tilde{Q}^{(3)}$ is the same as that for the unscaled quantities.
\begin{eqnarray}
\oint_{\Omega} d\vec{A}\cdot\langle\vec{\tilde{E}}^{(3)}_{\Omega,\Omega}(\vec{r})\rangle&=&
\langle \tilde{Q}^{(3)}_{\Omega,\Omega,\Omega}\rangle,
\end{eqnarray}
or for arbitrary hyper-surfaces
\begin{eqnarray}\label{eq:gauss_scaled}
\oint_{\Omega} d\vec{A}\cdot\langle\vec{\tilde{E}}^{(3)}_{\Omega,\Omega}(\vec{r})\rangle&=&
\int d\Omega_{\mu}\cdot\langle \tilde{j}^{\mu(3)}_{\Omega,\Omega}(\vec{r})\rangle.
\end{eqnarray}

The Kubo relations for the scaled quantities become:
\begin{eqnarray}
\langle \vec{\tilde{j}}^{(3)}_{\Omega\Omega}(\vec{r})\rangle&=&\sigma \langle\tilde{\vec{E}}^{(3)}_{\Omega,\Omega}(\vec{r})\rangle,\\
\nonumber
\langle\tilde{\vec{\mathcal{P}}}_{\Omega,\Omega}^{(3)}(\vec{r})\rangle&=&\kappa \langle\tilde{\vec{E}}^{(3)}_{\Omega,\Omega}(\vec{r})\rangle.
\end{eqnarray}
The relations for baryon conductivity and polarizabilty become
\begin{eqnarray}\label{eq:kuboscaled}
\langle\vec{j}_B(\vec{r})\rangle&=&\frac{\langle B_{\delta\Omega}\tilde{Q}^{(3)}_{\Omega,\Omega,\delta\Omega}\rangle}
{\langle[\tilde{Q}^{(3)}_{\Omega,\Omega,\delta\Omega\rangle}]^2\rangle}
\sigma\langle\vec{\tilde{E}}^{(3)}_{\Omega,\Omega}(\vec{r})\rangle,\\
\nonumber
\langle\vec{P}_B(\vec{r})\rangle&=&\frac{\langle B_{\delta\Omega}\tilde{Q}^{(3)}_{\Omega,\Omega,\delta\Omega}\rangle}
{\langle[\tilde{Q}^{(3)}_{\Omega,\Omega,\delta\Omega\rangle}]^2\rangle}
\kappa\langle\vec{\tilde{E}}^{(3)}_{\Omega,\Omega}(\vec{r})\rangle.
\end{eqnarray}
The ratio of correlators in the Eq.s (\ref{eq:kuboscaled}) would then not grow with $\Omega$, approaching a constant. Thus, the fluctuation of the baryon current in the center of a long flux tube would be proportional to the fluctuation of $\tilde{E}^{(3)}$ which would not continue to grow with $\Omega$. 

\subsection{Applicability of Baryon Transport Relations}

The expressions in Eq. (\ref{eq:kuboscaled}) for the induced baryon current or polarizability provide insight into how color charge couples to baryon transport. For a system with non-zero conductivity, the application of the field $\vec{\tilde{E}^{{(3)}}}$ induces a current. Combined with Gauss's laws, Eq.s (\ref{eq:gauss}) and (\ref{eq:gauss_scaled}), one see how a sub-volume with quadratic charge $\tilde{Q}^{(2)}$ provides an energy per unit length to a tube of area $A$, in Eq. (\ref{eq:tension}). Further, if the sub-volume has a cubic charge, $\tilde{Q}$, it provides a field $E^{(3)}$, through Gauss's law. This field can then induce a baryon current, or if there is no conductivity, it induces a baryon polarization. 

These relations offer particularly nice insight into the flux tubes. In that case one first chooses some point along the tube, and the subvolume $\Omega$ is then chosed as every part of a flux tube to one side of the bisection point. By assuming that the field is constant across a cross-sectional area that bisects a tube, one knows both the field, and thus the induced baryon transport. However, because this insight was based on Gauss's law, it does not necessarily translate well outside the paradigm of a flux tube. Even for a spherical volume one can imagine that the system has zero quadratic or cubic charge. But, that charge might have a dipole, or quadrupole moment. Thus, in general, it is challenging to calculate the field correlators outside the flux tube approximation. Even in the flux tube approximation, the Kubo relations are questionable because they are based on perturbation theory due to a slowly-changing applied field. But in reality, such fields exist only in extremely dynamic systems, such as during the first half fm/$c$ of a hadronic collision. Even inside a hadron, the relations are difficult to motivate because the quarks and antiquarks are not isolated to one side of some volume. Finally, one should not forget that perturbation theory fails in QCD for any length or time scales $\gtrsim$ 1 fm.

Despite the limits to the applicability mentioned above, the relations do provide some useful insight, even if that insight mainly provide some theoretical backbone to heuristic arguments. In the next section, compound flux tubes, i.e. tubes where the sub-volumes are in states of higher-order color multiplets, are presented. This phenomenology might have more promise for further development, and perhaps provide predictive capability.

\section{Compound Flux Tubes}\label{sec:compound}
\begin{figure}
\centerline{\includegraphics[width=0.5\textwidth]{figs/compoundtube.pdf}}
\caption{\label{fig:compoundtube}(color online)
The upper illustration show how a Gaussian surface isolates a flux tube into to parts. The color multiplets have interchanged values of $p$ and $q$. The energy density is defined by $\langle Q^{(2)}\rangle$, where the charge operators cover the left-hand side. 
}
\end{figure}

In Section \ref{sec:simpletubes}, simple flux tubes were discussed. The term ``simple'' was a tube with a quark at one end and an antiquark at the other. At any bisection point, the color multiplet to one side was either $(p=1,q=0)$ or $(p=0,q=1)$. Here, we discuss tubes where at the bisection point the multiplet to each might be described by a higher multiplet. For example havin a gluon to each side would be described by the octet $(p=1,q=1)$. By having multiple partons on a given side, any combination $(p,q)$ could be attained. The goal of this section is to investigate the possibilities of such combinations, and to understand how they might decay to the vacuum, which is a singlet $(p=0,q=0)$. The term ``compound'' tube refers to a tube with arbitrary $p$ and $q$. Similar considerations for more highly excited color flux tubes, or strings, have gone under the moniker of ``color ropes'' \cite{Sorge:1995dp,Goswami:2019mta}, and have been discussed as a means to increase nuclear stopping or enhance the production of heavy resonances.

Figure \ref{fig:compoundtube} illustrates a flux tube , where a Gaussian surface bisects the tube at some point as shown in the upper diagram. If the multiplet to one side is $(p,q)$ the multiplet describing the opposite side is $(q,p)$. Otherwise, the tube would not be in an overall color singlet. At the point of the bisection, the field-like color correlators are given by the color charges $Q^{(2)}$ and $Q^{(3)}$, which are simple functions of $p$ and $q$ defined in Eq. (\ref{eq:Q2Q3vspq}). The field energy is determined by $\langle Q^{(2)}\rangle_{\rm left}$ and the propensity to induce a baryon current is described by $\langle Q^{(3)}\rangle_{\rm left}$. If a $q\bar{q}$ pair tunnels through the vacuum, as shown in the lower diagram of Fig. \ref{fig:compoundtube}, the color multiplet of matter to the left of a point between the $q\bar{q}$ pair changes, and if it lowers $\langle Q^{(2)}\rangle_{\rm left}$, the energy density of the region between the $q\bar{q}$ is reduced. Eventually, $q\bar{q}$ pairs must be produced in such a way that one reaches a color singlet with $\langle Q^{(2)}\rangle_{\rm left}$.

For the field-like correlators at some point to vanish, partons must pass the point where the defining Gaussian surface intersects the flux tube. When a parton pass through the surface, the color state of the matter changes. If the initial multiplet is $(p_i,q_i)$ the available final states $(p_f,q_f)$ depend on whether the parton was a gluon, a quark or an antiquark.
\begin{eqnarray}\label{eq:pfqflist}
(p_i,q_i)+{\rm gluon}&\rightarrow& (p_f,q_f)=\left\{
\begin{array}{l}
(p_i+2,q_i-1)\\
(p_i+1,q_i-2)\\
(p_i-1,q_i+2)\\
(p_i-2,q_i+1)\\
(p_i+1,q_i+1)\\
(p_i-1,q_i-1)\\
(p_i,q_i)~~({\rm two~multiplets})
\end{array}
\right.\\
\nonumber
(p_i,q_i)+{\rm quark}&\rightarrow& (p_f,q_f)=\left\{
\begin{array}{l}
(p_i+1,q_i)\\
(p_i,q_i-1)\\
(p_i-1,q_i+1)
\end{array}
\right.\\
\nonumber
(p_i,q_i)+{\rm antiquark}&\rightarrow& (p_f,q_f)=\left\{\begin{array}{l}
(p_i,q_i+1)\\
(p_i-1,q_i)\\
(p_i+1,q_i-1)
\end{array}
\right.
\end{eqnarray}
\begin{figure}
\centerline{\includegraphics[width=0.75\textwidth]{figs/pq3.pdf}}
\caption{\label{fig:pq3}(color online)
Color multiplets are shown as a function of $(p,q)$ in the left-side panel. Starting from the singlet, ($p=0,q=0$), one can combine gluons to reach any of the multiplets denoted by black circles. These ``gluon-like'' states,  can also be reached by combining quarks and antiquarks if the difference of the number of quarks and antiquarks is a multiple of three. To reach a ``quark-like state'', one where $(p-q)/3$ has a remainder of +1 or -2, one can add a quark to a gluon-like state. One can reach an ``antiquark-like state'', one where the remainder of $(p-q)/3$ is -1 or 2, by adding an antiquark to a gluon-like state. The heavy black lines show the states available after adding either a gluon (heavy black line), an antiquark (light green line) or a quark (light dotted line). If gluon creation is possible, the most efficient way the system can reach a color singlet is to create gluon pairs, or perhaps one quark or antiquark depending on whether the original state is gluon-like, quark-like or antiquark-like. If gluonic steps are not permitted the most efficient decay mechanism is to add $p$ antiquarks and $q$ quarks. The right-side panel represents the same states, but the states are plotted as $\langle Q^{(2)}\rangle$ vs $\langle Q^{(3)}\rangle$. In this basis the energy per unit length is a function of $\langle Q^{(2)}\rangle$ and the field driving the baryon current is $\langle Q^{(3)}\rangle$.}
\end{figure}

If the initial tube is created by exchanging only gluons across the bisection point, the available multiplets must have $p-q$ in multiples of three. Even though such states were created by gluons passing through, both $p-q$ and $\langle Q^{(3)}\rangle_{\rm left}$ can be non-zero. Thus, they can induce baryon transport across the bisection point. However, if the decay involves gluon pairs, one can also return to the ground state without any quarks passing the bisection. If quarks and or antiquarks were initially exchanged across the bisection, one can also return to the ground state through the creation of gluon pairs if the difference in the quark number is a multiple of three. One can define a quark excess number as
\begin{eqnarray}
B_x&=\left\{\begin{array}{rl}
1,&N_{\rm quarks}-N_{\rm antiquarks}=\cdots,-5,-2,1,4,7\cdots\\
-1,&N_{\rm quarks}-N_{\rm antiquarks}=\cdots,-7,-4,-1,2,5\cdots\\
0, &N_{\rm quarks}-N_{\rm antiquarks}=\cdots,-6,-3,0,3,6\cdots
\end{array}\right.
\end{eqnarray}
If $B_x=1$ the multiplet is ``quark-like'', and at least one antiquark must pass into the left-hand side of the tube, or a quark must pass from left to right. Similarly, if $B_x=-1$ the multiplet is ``antiquark-like'', and at least one quark must pass the bisection from right to left, or an antiquark must pass from left to right. Additional groups of three quarks and three antiquarks are possible. This was expected, knowing that the net quark number on each side must be a multiple of three in the final state. If gluon pair creation is allowed, it represents the most efficient path to decaying the state in terms of the number of parton pairs created. In that case the net baryon number that passes the bisection point is $-1/3,1/3$ or zero.

If during the pair creation process the ability to create gluon pairs becomes negligible, returning to a color singlet requires $q-\bar{q}$ pair creation. The states available to each step are listed in Eq. (\ref{eq:pfqflist}) and are illustrated in Fig. \ref{fig:pq3}. If one couples a large number of partons, one can create multiplets with large value of $p$ and $q$. Multiplets with $p\approx q$ should be more likely because there are more ways in the creation process to couple a given number of partons to such a multiplet. Further, the degeneracy of a give multiplet, $d(p,q)=(p+1)(q+1)(p+q+2)/2$, increases both for higher $p+q$ and for higher $|p-q|$. Nonetheless, there is the possibility of creating multiplets with $p$ and $q$ significantly differing.

Because adding a quark (or antiquark) into a region during the decay process can lead to three possible multiplets, there are numerous possible paths from the original $(p_i,q_i)$ that lead to a color singlet. The most efficient path, that with the fewest partons passing the bisection point, is one where $p_i$ antiquarks and $q_i$ quarks cross into the left side. Thus, if through multiple-gluon exchange the initial state has $p_i-q_i=6$, and if the breakup stage strongly discourages the creation of gluon pairs compared to $q\bar{q}$ pairs, two baryons would transferred across the tube. 

Should the latter possibility, that gluon pair creation is discouraged during the decay process, one could imagine that baryon number might have to cross a large swath of rapidity for the system to recombine into color singlets. This does not, however imply that a single baryon moved across the fluxtube. Instead it is more that a large number of polarized $q\bar{q}$ pairs appeared. Just as polarizing a dielectric results in a surplus of electric charge on one side and a dearth of charge on the opposite side, these pairs appear without any actual movement of charge aside from the charges in individual pairs moving a small distance apart from one another to create the dipole density.

In this case where the tubes decay only via the tunneling of $q\bar{q}$ pairs, the number of quarks moving by a certain point is defined by the color multiplet, or more specifically by $p-q$, on either side of the point. The end of each tube is ultimately defined by a quark, antiquark or gluon, so the value of $p-q$ at the end of the tube is small. At each point in the tube where a quark or gluon was placed in the original state, the running value of $p-q$ changes, and the number of quarks or antiquarks passing these points also changes. Thus, at these points baryon number accumulates. For that reason if one thought of a tube as being orignally occupied by some number of partons spread along the tube, the decay process would likely lead to baryons or antibaryons congregating at those points. If one were to $p-q$ of the initial state vs. the position or spatial rapidity along the tube, $\eta$, and if the initial state was driven purely by gluons occupying various positions along the tube, the values of $p-q$ would either rise or fall by three, or stay the same, at each gluon's position. After the $q\bar{q}$ decays, antibaryons would appear at those positions where $p-q$ increased by three, and baryons would appear at those positions where it fell. 

Figure \ref{fig:2gluon} provides an illustration of how a tube might decay if the tube is driven by two charge multiplets, $(p=3,q=0)$ on the left edge and $(p=0,q=3)$ on the right edge.  These states might have been created by two gluons being exchanged between some target and some projectile. Assuming that the intermediate region is void of charge, the field-like correlators evaluated at some point in the intermediate region are defined by $p$ and $q$. For any position along the tube, the field energy will not disappear until come color charges have moved across that position. Then, the matter to the left, or right, of the specified position will then both be in color singlets. The number of quanta must be two or more. Two gluons crossing the position can accomplish the feat, as pictured in the upper illustration of Fig. \ref{fig:2gluon}. The first gluon can change each side to the multiplet $(p=1,q=1)$. The second gluon can then transform the sides into color singlets. If gluon pairs are readily produced from the field, or if gluons can be easily exchanged across the position, this should be the dominant mechanism for decay. No baryon current would be generated and no baryon anti-baryon pairs would be produced. Should gluonic modes not be easily excited, the decay of the tube would rely on the creation of $q-\bar{q}$ pairs. In that scenario, the most likely means of decay would require three quarks, or antiquarks, to pass any pre-specified position. As long as the net baryon charge moving to the left side is -1, one can divide the tube into two tubes, each in a color singlet. As shown in Fig. \ref{fig:2gluon} one side of the tube would have a baryon, while an anti-baryon would populate the opposite side. Of course, it is possible that both gluonic and $q\bar{q}$ decay modes might contribute. In that case, some positions on the tube might see two gluons cross in order to return both sides to a color singlet, while a different position might observe quarks passing through. In that case, the regions with quark movement would distill a baryon anti-baryon pair spanning the region. Other decay routes are also possible For example, a gluon might bring the system to the state $(p=1,q=1)$, followed by a quark and antiquark moving in the same direction, two quarks passing in opposite directions, or two antiquarks passing in opposite directions. No baryon charge would be distilled in this scenario.
\begin{figure}
\centerline{\includegraphics[width=0.8\textwidth]{figs/2gluon.pdf}}
\caption{\label{fig:2gluon}(color online)
An illustrative example of how a flux tube, whose initial state is characterized by a multiplet $(p=3,q=0)$ to the left, might decay. In the path (a), for any point along the tube, two gluons pass by. This restores the states to the left and right of that point to color singlets. The first gluon alters the left-side (as defined from the point) from $(p=3,q=0)$ to $(p=1,q=1)$. The right side also becomes a color octet. The second gluon brings both sides to a color singlet. No baryon number coalesces in this scenario. For path (b), only quarks pass any given point. This requires some combination of three quarks or antiquark to pass in order to create color singlets. Either three antiquarks pass the point towards the left, three quarks pass to the right, two quarks to the right and one antiquark to the left, or one quark to right and two antiquarks to the left. For any such change, the final state has an antibaryon on the left and a baryon to the right. Finally, in (c) an example is shown where in one region, three quarks pass by, while in another, two gluons pass. In this case, the final configuration has a baryon and antibaryon located at the opposite ends of the region where quarks and antiquarks were responsible for the $j^{(3)}$ current.
}
\end{figure}

From viewing Fig. \ref{fig:2gluon} one can see how a flux tube, even one created purely from a gluonic excitation, might decay is such a way to not only create baryon/anti-baryon pairs, but to significantly separate the baryon charge in rapidity. Assuming there is some characteristic length scale, $\lambda$, over which one cannot change from mode to the other, and assuming that the probability of being in the $q\bar{q}$ mode at any given point is $P_{q\bar{q}}$, the probability of having $N_{q\bar{q}}$  adjacent steps of size $\lambda$ all in the $q\bar{q}$ mode would be $P_{q\bar{q}}^{N_{q\bar{q}}}$, and the distributions of $q\bar{q}$ region sizes would fall off exponentially as a function of the distance $L$, i.e. $\propto \exp\{-(\ln(P_{q\bar{q}})L/\lambda\}$. The separation of balancing baryon/anti-baryon pairs would then be characterized by an exponential falloff. For the $(p=3,q=0)$ example illustrated in Fig. \ref{fig:2gluon} one could imagine that the color multiplet on the left side involved a baryon, which emitted two gluons to the right side. In that case, the three quarks could join with the three antiquarks distilling from the color field to produce mesons. The end result would be that the baryon number at the left would move to the boundary of the $q\bar{q}$ like decaying region. Thus, the baryon movement would also fall off exponentially, similar to the behavior expected from the simple flux tubes in Sec. \ref{sec:simple}, which involved merging tubes. However, even though the two process of baryon transport, one from a compound tube and the other from the merging of simple tubes, should both transport baryons from beam or target rapidities with an exponential falloff, there is no reason to think the two exponential scales would be the same.

Again, this mechanism for baryon-antibaryon separation would be  weaker for electric charge separation, vanishing in the limit that $u\bar{u},~d\bar{d}$ and $s\bar{s}$ tunneling became equally possible. Separation of balancing charges has indeed been measured, and has even been binned by whether the balancing charges are pions, kaons or protons. Indeed, the separation from protons has been observed to be significantly longer than that from pions in heavy ion collisions. However, much of that difference may simply arise from the fact that much of the electric charge is created late in the collision, through hadronization and decays. It would be interesting to experimentally analyze this class of correlations, known as charge balance functions, for $pp$ or $p\bar{p}$ collisions, focusing on baryon-antibaryon correlations.

\section{Summary}



\begin{acknowledgments}
This work was supported by the Department of Energy Office of Science through grant number DE-FG02-03ER41259.
\end{acknowledgments}

\bibliography{btransport}

\end{document}
